\documentclass[10pt]{beamer}
%\documentclass[10pt,handout]{beamer}
\usepackage[spanish]{babel}
% % \usepackage[backend=biber, style=authoryear-icomp]{biblatex}
\resetcounteronoverlays{exx}
\usepackage{mdframed}
\usepackage{tikz}
\usepackage{blindtext}
\usepackage{tipa}
% \usepackage{cgloss4e}
% \usepackage{gb4e}
% \usepackage{qtree}
\usepackage{cancel}
\usepackage{wrapfig}
\usepackage{soul}
\usepackage{enumerate}
\usepackage{longtable}
\graphicspath{ {../img/} } % declaramos donde estan las imagenes
\usepackage[labelformat=simple]{subcaption} % para varias imagenes juntas
\renewcommand\thesubfigure{(\alph{subfigure})}
\usepackage[utf8]{inputenc}
\usepackage{amsmath}
\usepackage{amsfonts} % simbolos como el I de matriz identidad
\usepackage{bm}
\usepackage{graphicx} % paquete para ver imagenes
\usepackage{setspace}
\usepackage[T1]{fontenc}
\usepackage{parskip}
\usepackage{color}
\usepackage{framed}

\usetheme{Copenhagen}
\definecolor{frenchblue}{rgb}{0.0, 0.45, 0.73} % ESTE!!!!
\definecolor{myblue1}{RGB}{35,119,189}
\definecolor{myblue2}{RGB}{95,179,238}
\definecolor{myblue3}{RGB}{129,168,207}
\definecolor{myblue4}{RGB}{26,89,142}

\setbeamercolor{block body}{bg=frenchblue!50}
\setbeamercolor*{structure}{fg=frenchblue,bg=blue}
\setbeamertemplate{frametitle}[default][center]
\setlength{\parskip}{12pt}
\useoutertheme{infolines} % me comia mucho espacio de la otra fgorma
\makeatother
\setbeamertemplate{footline}
{
  \leavevmode%
  \hbox{%
  \begin{beamercolorbox}[wd=.3\paperwidth,ht=2.25ex,dp=1ex,center]{author in head/foot}%
    \usebeamerfont{author in head/foot}\insertshortauthor
  \end{beamercolorbox}%
  \begin{beamercolorbox}[wd=.6\paperwidth,ht=2.25ex,dp=1ex,center]{title in head/foot}%
    \usebeamerfont{title in head/foot}\insertshorttitle
  \end{beamercolorbox}%
  \begin{beamercolorbox}[wd=.1\paperwidth,ht=2.25ex,dp=1ex,center]{date in head/foot}%
    \insertframenumber{} / \inserttotalframenumber\hspace*{1ex}
  \end{beamercolorbox}}%
  \vskip0pt%
}
\makeatletter
\setbeamertemplate{navigation symbols}{}
%\setbeameroption{show notes}
\setbeameroption{hide notes}


\usepackage{hyperref}

\title[Reporte quincenal]{Reporte quincenal}
\author[Matias Mazzanti]{Matias Mazzanti}


\institute{UBA-IBM(?)}
\date{05 de julio de 2022}

\titlegraphic{\includegraphics[,height=2cm,keepaspectratio]{../logo.pdf}     }

\begin{document}

\begin{frame}

\maketitle

\end{frame}


\section{Organización}
\begin{frame}
\frametitle{Roadmap}
 \begin{mdframed}[backgroundcolor=frenchblue!20]
  \textbf{Idea:} Iniciar 2023 sin ''burocracia'' (materias obligatorias) y con una buena base.
 \end{mdframed}

  \textbf{Julio}:  Final materia algoritmos y estructura de datos II (a confirmar).

  \textbf{Agosto}: 3/08 Final materia base de datos.

  \textbf{2do Cuatrimestre 2022} $\rightarrow$ dos partes: académica + papers.


\textbf{Verano 2023} $\rightarrow$ Cerrar gaps.


\end{frame}
%%%%%%%%%%%%%%%%%%%%%%%%%%%%%%%%%%%%%%%%%%%%%%%%%%%%%%%%%%%%%%%%%%%%%%%%%%%%%%%%%%%%%%%%%%%%%%%%%%%%

\begin{frame}
\frametitle{Roadmap Académico}

  \textbf{2do Cuatrimestre 2022} $\rightarrow$ Parte académica (ideas)


\begin{itemize}\itemsep-1em
  \item Curso \textbf{Criptografía} MIT: 24 clases $\times$ 1.2h. (De forma relajada)
  \item Curso presencial \textbf{Arquitecturas}: Orga2, 2 veces por semana 5h (''Oyente'').
  \item Lógica y computabilidad (\textbf{obligatorio}-plan doctoral), 2 veces por semana 3h (confirmar).
  \item Preparación para escuela de verano Barca (si se confirma).
\end{itemize}


  \textbf{Verano 2023} $\rightarrow$ Parte académica (ideas)

\begin{itemize}\itemsep-1em
  \item Final Lógica.
  \item Comunicaciones/Redes - Superficial.
  \item Sistemas Operativos - Superficial.
  \item Curso Algebra Online (solo si veo que se requiere)
\end{itemize}



\end{frame}
%%%%%%%%%%%%%%%%%%%%%%%%%%%%%%%%%%%%%%%%%%%%%%%%%%%%%%%%%%%%%%%%%%%%%%%%%%%%%%%%%%%%%%%%%%%%%%%%%%%%

\begin{frame}
\frametitle{Roadmap}


    \textbf{Express:} Libro HE and aplications (springer)
  \vspace{0.3cm}
  Ver con que intensidad.

 \begin{itemize}
    \item RSA'78 $\rightarrow$ RSD'78
    \item GM'82
    \item El-Gamal'85
    \item Pallier'99
    \item BGN'05
  \end{itemize}
  \vspace{0.3cm}

\centering 2 a 3 semanas.


\end{frame}

%%%%%%%%%%%%%%%%%%%%%%%%%%%%%%%%%%%%%%%%%%%%%%%%%%%%%%%%%%%%%%%%%%%%%%%%%%%%%%%%%%%%%%%%%%%%%%%%%%%%

\begin{frame}
\frametitle{Roadmap}
  \textbf{Papers/temas:} En orden 1 o 2  por reunion?

  \vspace{0.3cm}
  \begin{columns}
    \column{0.5\textwidth}
\begin{itemize}
  \vspace{0.3cm}\item FHE using Lattices[2009-Gentry].
  \vspace{0.3cm}\item Learning With Errors (LWE)[2009-Regev].
  \vspace{0.3cm}\item Ring-LWE[2013-Lyu-Peikert-Regev].
  \vspace{0.3cm}\item RGSW[2013-Gentry-Sahai-Waters]
  \vspace{0.3cm}\item Boostraping with polynomial error [2014-Alperin-Sheriff].
\end{itemize}

    \column{0.5\textwidth}
\begin{itemize}
  \vspace{0.3cm}\item Fast Boostraping [2016-Chillotti-Gama-etc].
  \vspace{0.3cm}\item CKKS [2017].
  \vspace{0.3cm}\item Number Theory Transform (NTT).
  \vspace{0.3cm}\item Cryptosystems: FHEW[2015-Ducas-Micciancio] y TFHE[2018-Chilliotti-Gama-etc].
\end{itemize}

  \end{columns}

    \vspace{0.5cm}

    \centering
    \textbf{2 a 3 meses}

\end{frame}

%%%%%%%%%%%%%%%%%%%%%%%%%%%%%%%%%%%%%%%%%%%%%%%%%%%%%%%%%%%%%%%%%%%%%%%%%%%%%%%%%%%%%%%%%%%%%%%%%%%%

\begin{frame}
\frametitle{Roadmap}
  \textbf{Siguiente paso} $\rightarrow$ Verano?

  Papers/temas: Arquitecturas y mas (queda por definir)
\begin{itemize}
    \item CPU-GPU-FHE[2020-Morshed-AlAziz-Mohammed].
    \item Papers recomendados en mail.
    \item Investigar más al respecto de arquitecturas.
\end{itemize}



\end{frame}

%%%%%%%%%%%%%%%%%%%%%%%%%%%%%%%%%%%%%%%%%%%%%%%%%%%%%%%%%%%%%%%%%%%%%%%%%%%%%%%%%%%%%%%%%%%%%%%%%%%%
\section{Paper}
\begin{frame}
\frametitle{FHE...}
  Esquema de encriptacion clave publica: $\varepsilon$

  Texto plano $\pi \in P$, texto cifrado $\psi \in \zeta$, $C$ circuitos, $\lambda$ parámetro de seguridad.
  \vspace{0.3cm}

\begin{columns}
  \column{0.5\textwidth}
     \textbf{KeyGen}$_\varepsilon(\lambda) \rightarrow (sk,pk)$

  \vspace{0.3cm}
  \textbf{Encrypt}$_\varepsilon(pk, \pi) \rightarrow \psi$

  \vspace{0.3cm}
  \textbf{Decrypt}$_\varepsilon(sk, \psi) \rightarrow \pi$

  \vspace{0.3cm}
  \textbf{Evaluate}$_\varepsilon(pk, C, \Psi) \rightarrow$ \textbf{Encrpyt}$_\varepsilon(pk, C(\pi_1,...\pi_t))$

  \vspace{0.3cm}
  $\Psi = <\psi_1,...,\psi_t>$

  \column{0.5\textwidth}
      FHE: Cualquier circuitos.

  \vspace{0.3cm}
  Privacidad:  Estadísticamente indistinguibles
  \vspace{0.3cm}

  \textbf{Encrpyt}$_\varepsilon(pk, C(\pi_1,...\pi_t)) \approx  \textbf{Evaluate}_\varepsilon(pk, C, \Psi)$


\end{columns}

\end{frame}
%%%%%%%%%%%%%%%%%%%%%%%%%%%%%%%%%%%%%%%%%%%%%%%%%%%%%%%%%%%%%%%%%%%%%%%%%%%%%%%%%%%%%%%%%%%%%%%%%%%%
\section{Paper}
\begin{frame}
\frametitle{Gentry...}

  Empezando por el ''principio''  $\rightarrow$ \textbf{FHE Using Ideal Lattices} 2009 (y su tesis)

  Hasta entonces:
  \begin{itemize}
    \item  $\psi$ crecía exponencialmente con la profundidad multiplicativa (y aveces con la suma) del circuito.
    \item O solo aceptaban un conjunto de circuitos.
  \end{itemize}
\end{frame}


%%%%%%%%%%%%%%%%%%%%%%%%%%%%%%%%%%%%%%%%%%%%%%%%%%%%%%%%%%%%%%%%%%%%%%%%%%%%%%%%%%%%%%%%%%%%%%%%%%%%

\section{Lo siguiente}
\begin{frame}
\frametitle{}


\end{frame}


%%%%%%%%%%%%%%%%%%%%%%%%%%%%%%%%%%%%%%%%%%%%%%%%%%%%%%%%%%%%%%%%%%%%%%%%%%%%%%%%%%%%%%%%%%%%%%%%%%%%

\end{document}

