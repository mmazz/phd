\documentclass[10pt]{beamer}
%\documentclass[10pt,handout]{beamer}
\usepackage[spanish]{babel}
% % \usepackage[backend=biber, style=authoryear-icomp]{biblatex}
\resetcounteronoverlays{exx}
\usepackage{mdframed}
\usepackage{tikz}
\usepackage{blindtext}
\usepackage{tipa}
% \usepackage{cgloss4e}
% \usepackage{gb4e}
% \usepackage{qtree}
\usepackage{cancel}
\usepackage{wrapfig}
\usepackage{soul}
\usepackage{enumerate}
\usepackage{longtable}
\graphicspath{ {../img/} } % declaramos donde estan las imagenes
\usepackage[labelformat=simple]{subcaption} % para varias imagenes juntas
\renewcommand\thesubfigure{(\alph{subfigure})}
\usepackage[utf8]{inputenc}
\usepackage{amsmath}
\usepackage{amsfonts} % simbolos como el I de matriz identidad
\usepackage{bm}
\usepackage{graphicx} % paquete para ver imagenes
\usepackage{setspace}
\usepackage[T1]{fontenc}
\usepackage{parskip}
\usepackage{color}
\usepackage{framed}
\usetheme{Copenhagen}
\definecolor{frenchblue}{rgb}{0.0, 0.45, 0.73} % ESTE!!!!
\definecolor{myblue1}{RGB}{35,119,189}
\definecolor{myblue2}{RGB}{95,179,238}
\definecolor{myblue3}{RGB}{129,168,207}
\definecolor{myblue4}{RGB}{26,89,142}

\setbeamercolor{block body}{bg=frenchblue!50}
\setbeamercolor*{structure}{fg=frenchblue,bg=blue}
\setbeamertemplate{frametitle}[default][center]
\setlength{\parskip}{12pt}
\useoutertheme{infolines} % me comia mucho espacio de la otra fgorma
\makeatother
\setbeamertemplate{footline}
{
  \leavevmode%
  \hbox{%
  \begin{beamercolorbox}[wd=.3\paperwidth,ht=2.25ex,dp=1ex,center]{author in head/foot}%
    \usebeamerfont{author in head/foot}\insertshortauthor
  \end{beamercolorbox}%
  \begin{beamercolorbox}[wd=.6\paperwidth,ht=2.25ex,dp=1ex,center]{title in head/foot}%
    \usebeamerfont{title in head/foot}\insertshorttitle
  \end{beamercolorbox}%
  \begin{beamercolorbox}[wd=.1\paperwidth,ht=2.25ex,dp=1ex,center]{date in head/foot}%
    \insertframenumber{} / \inserttotalframenumber\hspace*{1ex}
  \end{beamercolorbox}}%
  \vskip0pt%
}
\newcommand{\floor}[1]{\lfloor #1 \rfloor}

\makeatletter
\setbeamertemplate{navigation symbols}{}
%\setbeameroption{show notes}
\setbeameroption{hide notes}


\usepackage{hyperref}

\title[Reporte quincenal]{Reporte quincenal}
\author[Matias Mazzanti]{Matias Mazzanti}


\begin{document}

%%%%%%%%%%%%%%%%%%%%%%%%%%%%%%%%%%%%%%%%%%%%%%%%%%%%%%%%%%%%%%%%%%%%%%%%%%%%%%%%%%%%%%%%%%%%%%%%%%%%

\begin{frame}
\frametitle{CKKS: repaso}
Sea $z\in \mathbb{C}^{N/2}$, codificarlo en un polinomio  $m(X)\in \mathcal{R} = \mathbb{Z}[X]/(X^N+1)$
utilizando el encaje canónico.
\pause

$ \mathbb{C}^{N/2} \to \mathcal{R} =\mathbb{Z}[X]/(X^N+1)$.
\pause

Sea $N$, múltiplo de 2, $M=2N$ y el
 \textbf{polinomio cíclico} $\Phi_M(X)= X^N+1$.
\pause

Donde sus \textbf{raíces} de la unidad son: $\xi_M=(e^{2i\pi/M})^k$, $k\in \mathbb{Z}<M$ y coprimos de $M$.
\pause

Dada la codificación, decodificamos evaluando $m(X)$ con las raices de la unidad.

\end{frame}
%%%%%%%%%%%%%%%%%%%%%%%%%%%%%%%%%%%%%%%%%%%%%%%%%%%%%%%%%%%%%%%%%%%%%%%%%%%%%%%%%%%%%%%%%%%%%%%%%%%%

\begin{frame}
\frametitle{CKKS: repaso}
Para la codificación primero expandimos nuestro vector $\in  \mathbb{C}^{N/2}$ a $ \mathbb{C}^{N}$
agregando sus complejos conjugados.
\pause

Luego projectamos a la base del encaje canónico y a su vez redondeamos los coeficientes.
(isomorfismo entre el vector y el polinomio.)
\pause

Por que se usan polinomios? \pause permite encriptar un vector de multiples valores en un único cifrado
y tiene mejor rendimiento para operar (DFT).

\pause
Ademas, vimos que se multiplica a $z$ por un factor $\bigtriangleup>0$  para evitar
que el redondeo elimine alguno de los números significativos.

\end{frame}
%%%%%%%%%%%%%%%%%%%%%%%%%%%%%%%%%%%%%%%%%%%%%%%%%%%%%%%%%%%%%%%%%%%%%%%%%%%%%%%%%%%%%%%%%%%%%%%%%%%%

\begin{frame}
\frametitle{CKKS: continuación}
CKKS utiliza los problemas dificiles de \textit{Ring-Learning With Errors} (RLWE).
\pause

Con esto se genera la clave publica $p$ y secreta $s$:

$p = (b,a) = (-a.s+e,a)$

Donde a,e y s $\in Z_q[X]/(X_N+1)$.
\pause

$a$ esta muestreado uniformemente y $e$ es un pequeño error polinomial.
\pause

\end{frame}
%%%%%%%%%%%%%%%%%%%%%%%%%%%%%%%%%%%%%%%%%%%%%%%%%%%%%%%%%%%%%%%%%%%%%%%%%%%%%%%%%%%%%%%%%%%%%%%%%%%%


\begin{frame}
\frametitle{CKKS: encriptación}
Sea $\mu\in Z_q[X]/(X_N+1)$ (mensaje codificado)
\pause

La encriptación usando $p = (b,a) = (-a.s+e,a)$ viene dada por
\begin{align*}
 Encrypt(\mu,p) =  c = (\mu,0)+p = (\mu+b,a) =(\mu -a.s+e,a) = (c_0,c_1)
\end{align*}
\pause

La decriptación de $c$ usando $sk = (1, s)$ viene dada por
\begin{align*}
 Decrypt(c,sk)= c*sk =  c_0+c_1.s = \mu-a.s+e+a.s = \mu+e \approx \mu
\end{align*}
\end{frame}
%%%%%%%%%%%%%%%%%%%%%%%%%%%%%%%%%%%%%%%%%%%%%%%%%%%%%%%%%%%%%%%%%%%%%%%%%%%%%%%%%%%%%%%%%%%%%%%%%%%%
\begin{frame}
\frametitle{CKKS: suma}
Sean dos mensajes $\mu$ y $\mu'$ encriptados como $c$ y $c'$ respectivamente.

\begin{align*}
\action<+->  { CAdd(c,c') &= c_{add} = c + c' = (c_0,c_1) + (c'_0, c'_1) }\\
           \action<+->  {   &= ((c_0+c_0') , (c_1+c_1')) =(c_{add,0}, c_{add,1})}
\end{align*}
\pause

Al desencriptarlo:
\begin{align*}
  \action<+->  { Decrypt(c_{add},sk) &= c_{add}*sk =  c_{add,0}+ c_{add,1}.s = (c_0+c'_0) + (c'_1 + c'_1).s }\\
  \action<+->  {                &= (c_0+c_1.s) +  (c'_0+c'_1.s) =  Decrypt(c,s)+Decrypt(c',s)}\\
  \action<+-> {                &= (\mu +e) + (\mu'+ e) = \mu + \mu'+ 2e = \approx \mu + \mu'}
\end{align*}
\end{frame}
%%%%%%%%%%%%%%%%%%%%%%%%%%%%%%%%%%%%%%%%%%%%%%%%%%%%%%%%%%%%%%%%%%%%%%%%%%%%%%%%%%%%%%%%%%%%%%%%%%%%
\begin{frame}
\frametitle{CKKS: Multiplicación cifrado-plano}
Sea $\mu$ un mensaje encriptado y $\mu'$ un dato codificado (sin necesidad de encritarlo)

\begin{align*}
 CMult(\mu,\mu')=  c_{mult} =  \mu.\mu'= (c_0,c_1) . \mu' =  (c_0.\mu',c_1.\mu')
\end{align*}
\pause

Al desencriptarlo:

\begin{align*}
  \action<+-> { DecryptMult(c_{mult},sk) &= c_0.\mu'+c_1.\mu'.s =  \mu'.(c_0,c_1.s)} \\
    \action<+-> { &= \mu'.(\mu+e) = \mu'.\mu + \mu'.e \approx \mu'.\mu}
\end{align*}

\action<+-> { Vemos que definimos una nueva función para desencriptar la multiplicación. Porque?}
\end{frame}

%%%%%%%%%%%%%%%%%%%%%%%%%%%%%%%%%%%%%%%%%%%%%%%%%%%%%%%%%%%%%%%%%%%%%%%%%%%%%%%%%%%%%%%%%%%%%%%%%%%%
\begin{frame}
\frametitle{CKKS: Multiplicación cifrado-cifrado}
Sean dos mensajes $\mu$ y $\mu'$ encriptados como $c$ y $c'$ respectivamente.

Recordamos que $Decrypt(c,sk)=c0+c1.s$\pause

Lo que queremos:
\begin{align*}
\action<+-> { DecryptMult(CMult(c,c'),sk)&=Decrypt(c,sk).Decrypt(c',s)} \\
\action<+-> {                           &=(c_0+c_1.s).(c'_0+c'_1.s)}\\
\action<+-> {                           &=c_0.c'_0+(c_0.c'_1+c'_0.c_1).s+c_1.c'_1.s^2}\\
\action<+-> {                           &=d_0+d_1.s+d_2.s^2}
\end{align*}
\action<+-> {  con $ d_0=c_0.c'_0$, $d_1=(c_0.c'_1+c'_0.c_1)$, $d_2=c_1.c'_1$}
\pause

  La desencriptación de la multiplicación de dos cifrados es un polinomio de orden dos
  evaluado en la llave secreta $s$.
\end{frame}
%%%%%%%%%%%%%%%%%%%%%%%%%%%%%%%%%%%%%%%%%%%%%%%%%%%%%%%%%%%%%%%%%%%%%%%%%%%%%%%%%%%%%%%%%%%%%%%%%%%%
\begin{frame}
\frametitle{CKKS: Multiplicación cifrado-cifrado}
Entonces tenemos definimos:
\begin{align*}
  \action<+-> {   &CMult(c,c')=c_{mult}=(d_0,d_1,d_2)=(c_0.c'_0,c_0.c'_1+c'_0.c_1,c_1.c'_1)} \\
  \action<+-> {   &DecryptMult(c_{mult},s)=d_0+d_1.s+d_2.s^2}
\end{align*}
\pause

Esto no es de utilidad.
Por cada operación de estas, el cifrado aumenta el grado del polinomio!
\pause

La multiplicación de dos cifrados queda representado como la suma de 3 polinomios.

Si volvemos a multiplicar nos quedara en 5, luego 9, y siguiendo con crecimiento exponencial.
\end{frame}
%%%%%%%%%%%%%%%%%%%%%%%%%%%%%%%%%%%%%%%%%%%%%%%%%%%%%%%%%%%%%%%%%%%%%%%%%%%%%%%%%%%%%%%%%%%%%%%%%%%%
\begin{frame}
  \frametitle{CKKS: Relinealización}
Tenemos $CMult(c,c')=(d_0,d_1,d_2)$.

El truco es computar de otra forma el termino cuadratico $d_2.s^2$ de forma que la
desencriptación luego me quede lineal.
\pause

Es decir, quiero encontrar dos polinomios $ (d'_0,d'_1)=c_{mult} = (d_0, d_1, d_2)$ tal que:
\begin{align*}
  Decrypt((d'_0,d'_1),s)&=d'_0+d'_1.s=d_0+d_1.s+d_2.s^2\\
 &=Decrypt(c,s).Decrypt(c',s)
\end{align*}

Y aplicaremos la relinealización luego de cada multiplicación.

Para ello nos generaremos una clave de evaluación (proxima reunion).
\end{frame}
%%%%%%%%%%%%%%%%%%%%%%%%%%%%%%%%%%%%%%%%%%%%%%%%%%%%%%%%%%%%%%%%%%%%%%%%%%%%%%%%%%%%%%%%%%%%%%%%%%%%

\begin{frame}
  \frametitle{CKKS: próxima}

\begin{itemize}
  \item Definir la función de relinealizar.
  \item Rescaleo: Maneja el crecimiento del ruido debido a multiplicar el factor de escaleo de la codificación $\bigtriangleup$ por si mismo.
    \pause
  \item Utilización del teorema chino del remante para optimizar el rescaleo (computos con
    polinomeos muy grandes).
\end{itemize}
\pause

Queda pendiente:
\begin{itemize}
  \item Utilización de DFT en la codificación en OpenFHE.
  \item Ajustar la implementar en Python (No me funciona la multiplicación).
    \pause
  \item Entender la primer variante de Bootstrapping en CKKS.
\end{itemize}


\end{frame}


\end{document}

