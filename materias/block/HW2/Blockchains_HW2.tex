\documentclass[12pt,addpoints,answers]{exam}
\usepackage[utf8]{inputenc}
\usepackage{amsmath,amsfonts,amssymb,amsthm}
\usepackage[margin=1in]{geometry}
\usepackage{mathtools}
\usepackage{hyperref}
\usepackage{fullpage}
\usepackage{microtype}
\usepackage{xspace}
\usepackage[svgnames]{xcolor}
\usepackage[sc]{mathpazo}
\usepackage{enumitem}
\usepackage{bm}

\pagestyle{head}

%----------Header--------------------%
\def\course{{\sc Foundations and Applications of Blockchains}}
\def\term{Depto. de Computaci\'{o}n, $2^{\textrm{do}}$ Cuatrimestre 2023}
\def\prof{Lecturer: Juan Garay}
\newcommand{\handout}[5]{
   \renewcommand{\thepage}{\arabic{page}}
   \begin{center}
   \framebox{
      \vbox{
    \hbox to 5.78in { \hfill \large{\course} \hfill }
    \vspace{2mm}
%    \hbox to 5.78in { \hfill \large{\prof} \hfill }
%       \vspace{2mm}
       \hbox to 5.78in { {\Large \hfill \textbf{#5}  \hfill} }
       \vspace{2mm}
       \hbox to 5.78in { \term \hfill \emph{#2}}
       \hbox to 5.78in { {#3 \hfill \emph{#4}}}
      }
   }
   \end{center}
   \vspace*{4mm}
}
\newcommand{\hw}[4]{\handout{#1}{#2}{#3}{#4}{Homework #2}}

% -- For ignoring stuff -- %
\newcommand{\ignore}[1]{}

\begin{document}

%----Specs: change accordingly-----%
\newif\ifstudent % comment out false
\studenttrue
% \studentfalse

\def\hwnum{1}
\def\issuedate{14/9/23}
\def\duedate{25/9/23, 3:30pm} %
\def\yourname{} % put your name here
%------------------------------%

\ifstudent
\hw{\hwnum}{\issuedate}{Student: \yourname}{Due: \duedate}%
\else
\hw{\hwnum}{\issuedate}{\prof}{Due: \duedate}%
\fi

% \ignore{

\noindent \textbf{Instructions}

\begin{itemize}
    \item The way to deliver your solution will be explained in the following classes.

     If you are proficient with \LaTeX, you may also typeset your submission and submit in PDF format. To do so, uncomment the ``\%\textbackslash begin\{solution\}'' and ``\%\textbackslash end\{solution\}'' lines and write your solution between those two command lines.

      \item Your solutions will be graded on \emph{correctness} and
    \emph{clarity}. You should only submit work that you believe to be
    correct.
    % , and you will get significantly more partial credit if you     clearly identify the gap(s) in your solution.

    \item You may collaborate with others on Questions 1-3 of this problem set. However, you
must write up your own solutions and list your collaborators and any external
sources for each problem. Be ready to explain your solutions orally to a member of
the course staff if asked. You may not collaborate with others on Question 4


\end{itemize}

\noindent This homework contains \numquestions\ questions,
% \numpages\ pages
for a total of \numpoints \ points.
% and \numbonuspoints\ bonus points.

%\medskip
\newpage

%} %IGNORE

\begin{questions}


\question We saw in class the notion of digital signatures and their security properties, existential
unforgeability being an important one.

\begin{parts}
    \part[5] Describe the purpose of a Public-Key Infrastructure (PKI). Can the
security properties of a digital signature be guaranteed without a PKI? Elaborate.

    \begin{solution} %Uncomment and type your solution here
    El proposito de una PKI en firmas digitales es poder usar la llave secreta para la firma
        y la llave publica para validar.
        Esto es de gran ayuda cuando tenemos un servidor de confianza donde puede
        confirmar de quien es cada llave publica.
        Teniendo esto podemos validar algo firmado de forma facil y eficiente.
        De todas formas se puede tener las propiedades de seguridad de una firma digital
        sin una PKI siempre y cuando el consenso de  n participantes es mayor que 3t (Bor96)

    \end{solution}

    \part[5] Bitcoin transactions use the ECDSA signature scheme. Does Bitcoin
assume a PKI? If not, reconcile with the above argument.

    \begin{solution} %Uncomment and type your solution here
        Bitcoin no asume  un PKI.
    \end{solution}


\end{parts}

\newpage

    \question[10] Refer to Algorithm 4 (Main Loop) in [GKL15]. We saw in class that
blockchains would start from a “genesis” block, which would provide an unpredictable string to start mining from.
    Yet, notice that in Algorithm 4 mining starts from the empty string $(C \leftarrow \epsilon)$.
    How come? Explain why this works

\begin{solution} %Uncomment and type your solution here
Esto funciona por que una string vacia tambien puede ser hasheada, entonces no genera
ninugn problema.
\end{solution}



\newpage

\question[10] Refer to Algorithm 1 (validate) in [GKL15], which implements the chain validation
predicate.


 \begin{parts}
    \part[5] Rewrite validate so that it starts checking from the beginning of the chain.
    \begin{solution} %Uncomment and type your solution here
        Suponiendo la funcion que permita sacar un bloque desde el principio,
        y que podemos agarrar el primer bloque de una cadena, el algoritmo
        quedaria igual con una salvedad.
        De que ahora tenemos que ir a agarrar otro bloque (el siguiente) para poder agarrar el puntero del bloque anterior (parametro s),
        para poder validar de que el hash del bloque que tengo es igual a la infomracion del parametro s guardado en el siguiente bloque.
    \end{solution}
    \part[5] Discuss pros and cons of this approach, compared to Algorithm 1’s.
    \begin{solution} %Uncomment and type your solution here
        Uno de las contras que le encuentro es que ahora tenemos que estar agarrando dos bloques para poder ver si uno apunta al otro.
        Que en el caso original como vamos desde el final para el principio simplemente nos guardamos la informacion del parametro s
        del bloque en que estamos para que el siguiente iteracion lo usemos para comparar con el hash del bloque actual.
    \end{solution}
 \end{parts}



\newpage

\question
Smart contract programming: Matching Pennies. This assignment will focus on
writing your own smart contract to implement the Matching Pennies game. The
contract should allow two players (A, B) to play a game of Matching Pennies at any
point in time. Each player picks a value of two options—for example, the options
might be \{0, 1\}, \{‘a’, ‘b’\}, \{True, False\}, etc. If both players pick the same value, the
first player wins; if players pick different values, the second player wins. The winner
gets 0.1 ETH as reward. After a game ends, two different players should be able to
use your contract to play a new game.

Example: Let A, B be two players who play the game, each with 1 ETH. A picks 0
and B picks 0, so A wins. After the game ends, A’s balance is 1.1 ETH (perhaps minus
some gas fees, if necessary).

You should implement the smart contract and deploy it on the course’s Goerli Testnet.
Your contract should be as secure, gas efficient, and fair as possible. After deploying
your contract, you should engage with other student’s contract and play a game
on his/her contract. Before you engage with a fellow student’s smart contract, you
should evaluate their code and analyze its features in terms of security and fairness
(refer to Lectures 07 and 08). You should provide:




 \begin{parts}

 \part[10] The code of your contract, together with a detailed description of the
high-level decisions you made for the design of your contract, including:
\begin{itemize}
    \item Who pays for the reward of the winner?
    \item  How is the reward sent to the winner?
    \item  How is it guaranteed that a player cannot cheat?
    \item What data type/structure did you use for the pick options and why?
\end{itemize}
\begin{solution} %Uncomment and type your solution here
    1) el contrato es el que paga ya que antes de jugar el contrato se queda con la apuesta de ambos.

    2) La recompensa tiene que ser pedida por cada usuario.
    Basicamente el contrato tiene los depositos originales de cada jugador mas menos lo que
    fueron ganando o perdiendo.
    Entonces cuando quieren solicitan sacar la plata y listo.
    Esto evita posibles problemas de seguiridad.

    3) Cada usuario elije una opcion, pero envia el hash de la opcion con un numero elegido por el.
    Una vez que ambos la elijen pueden pedir abrirla, dando por separado la opcion y el numero.
    Entonces el contrato verifica que el hash de estas dos cosas es la opcion que puso el usuario
    y cuando el segundo jugador tambien abre su opcion el contrato verifica quien gano.

    4) Use un diccionario address jugador con un bytes32 que contiene el hash de la opcion.
    Use dos diccionarios, uno con la opcion hasheada con el numero random y luego otra
    donde ya se encuentra la opcion revelada.
    Por lo explicado en el anterior de esta forma un usuario para que la informacion de la eleccion
    del contricante este en la blockchain de forma revelada el otro jugador tambien tuvo que haber elejido.
    Y si quiere sacar la plata por que vio que el otro gano, el contrato asume que pierde directamente
    y le da la ganancia al otro jugador.
\end{solution}

\part[5]A detailed gas consumption evaluation of your implementation, including:

\begin{itemize}
    \item The cost of deploying and interacting with your contract.
    \item  Whether your contract is fair to both players, including whether one player has to pay more gas than the other and why.
    \item Techniques to make your contract more cost efficient and/or fair
\end{itemize}

\begin{solution} %Uncomment and type your solution here
El costo de deploying es de 1942115 Wei
\end{solution}

\part[5]A thorough list of potential hazards and vulnerabilities that may occur
in your contract; provide a detailed analysis of the security mechanisms you use
to mitigate such hazards.

\begin{solution} %Uncomment and type your solution here
    No queda definido que pasa si hacen las elecciones y antes de revelar retirar la plata.
    Esa plata queda en el contrato.
    Mas que eso no creo que haya otra vulnerabilidad ni que pueda ser explotada.
    Mas que eso creo que no.
    Creo que me asegure que no haya reentrada, ni que un usuario pueda hacer que
    el otro pierda ethers por haber apostado y retirado.
\end{solution}


\part[5] A description of your analysis of your fellow student’s contract (along
with relative code snippets of their contract, where needed for readability),
including:

\begin{solution} %Uncomment and type your solution here
Revise el contrato del compañero Luciano Tarsia,
\end{solution}
\part[5]
\begin{itemize}
    \item Any vulnerabilities discovered?
    \item How could a player exploit these vulnerabilities to win the game
\end{itemize}

\begin{solution} %Uncomment and type your solution here
 Por lo que veo el contrato no genera ninugn tipo de seguridad para la eleccion del
    usuario que primero juega.
    Es decir el ultimo que hace su eleccion puede ver en la blockchain la eleccion del primero
    y realizar la eleccion correcta para siempre ganar.
    Esto claramente genera el exploit de poder siempre ganar.
\end{solution}
     \part[5] The transaction history of an execution of a game on your contrac
\begin{solution} %Uncomment and type your solution here
    \begin{itemize}
        \item Hago un deposito de 1000Gwei (cambie la cantidad requerida ya que nadie en la testnet tenemos 1ether...), 0x9b213f6b15355d5e35acbdf6a4d6778c262b96fecc6c32bebc92524ea43b1cc4
        \item Me fijo el hash de lo la eleccion, Ejemplo: HEAD:  0x4845414400000000000000000000000000000000000000000000000000000000
        \item Luego usando hashChoice pongo el hash de recien mas un numero que quiero y me hashea,
            Ejemplo con el numero 14 y da: 0xbbf2ff047966babbabfc43a5f8013fbd71a5ab35692a9453601380e7e469a4f2
        \item pego esto en play, 0x2a282564ff34872df40e2a6fe18c445f276901204a602a359dc06394315aaeaf.
        \item Ahora tengo que esperar a que el otro jugador elija su eleccion.
        \item Ahora pego el hash original de mi eleccion mas el numero que elegi en evaluateChoice.
        \item Ahora puedo volver a jugar o retirar el dinero con withdraw.
        \item En este caso hice withdraw, 0x7fcea6438092c1b20cee25c2d60b24ab29a5c045eb27440b2dfec82fab99e4ad
    \end{itemize}
\end{solution}

\end{parts}
\begin{verbatim}

// SPDX-License-Identifier: GPL-3.0
pragma solidity ^0.8.0;


contract CoinFlip
{
    bytes32 public constant HEAD = "HEAD";
    bytes32 public constant TAIL = "TAIL";

    uint public num_players = 0;
    address[2] public players;
    mapping(address => bytes32) public choices;
    mapping(address => bytes32) public choicesRevealed;

    mapping(address => uint256) private userBalances;


    // Pago inicial
    function deposit(uint256 deposito) payable public
    {
        // que no haya mas de 2 jugadores
        require(num_players < 2, "Max players, wait for one of them end playing");
        require(msg.value >= 100 gwei, "Not enough funds"); // fondos... 1 eth
        if(players[0]==address(0))
        {
            players[0]=msg.sender;
        }
        else
        {
            players[1]=msg.sender;
        }
        userBalances[msg.sender] += deposito;
        num_players += 1;
    }

    function balanceOf(address account) public view returns (uint256)
    {
        return userBalances[account];
    }

    function hashChoice(bytes32 choice, uint64 randomness) external pure returns(bytes32)
    {
        return keccak256(abi.encodePacked(choice, randomness));
    }

// sol reentrancy
    function withdrawBalance() public
    {
        // evito que un jugador pueda no revelar su opcion y el otro si y retirar si sabe que perdio.
        // basicamente se lo doy como perdida y retira. Pero le doy al otro usuario su ganancia
        if(msg.sender == players[0])
        {
            if(choicesRevealed[players[1]]!= bytes32(0))
            {
                userBalances[players[1]] += 200;
            }
        }
        if(msg.sender == players[1])
        {
            if(choicesRevealed[players[0]]!= bytes32(0))
            {
                userBalances[players[0]] += 200;
            }
        }
        uint amountToWithdraw = userBalances[msg.sender];
        userBalances[msg.sender] = 0;
        delete userBalances[msg.sender];
        num_players -= 1;
        // para evitar que uno revele su respuesta y el otro retire todo
        choicesRevealed[players[0]] = bytes32(0);
        choicesRevealed[players[1]] = bytes32(0);
        choices[players[0]] = bytes32(0);
        choices[players[1]] = bytes32(0);
        payable(msg.sender).transfer(amountToWithdraw);
    }

    function play(bytes32 choice) payable public
    {
        require(choices[msg.sender] == bytes32(0), "You had make your choice");
        require(userBalances[msg.sender] >= 100, "Not enough funds in the contract, make a new deposit");
        userBalances[msg.sender] -= 100;
        choices[msg.sender] = choice;
    }

    function evaluateChoice(bytes32 choice, uint64 randomness) external
    {
        // que los dos ya hallan elegido
        require(choices[players[0]] != bytes32(0), " Player 0 has not make a choice");
        require(choices[players[1]] != bytes32(0), " Player 1 has not make a choice");
        require(keccak256(abi.encodePacked(choice, randomness)) == choices[msg.sender], "Your choice and randomness do not correspond to your previus selection");
        choicesRevealed[msg.sender] = choice;
        // Si ya jugaron los dos directamente se ejecuta el evaluate
        if((msg.sender==players[0]) && (choicesRevealed[players[1]]!= bytes32(0)))
        {
            evaluate();
        }
        else if((msg.sender==players[1]) && (choicesRevealed[players[0]]!= bytes32(0)))
        {
            evaluate();
        }
    }

    function evaluate() internal
    {
        require(choicesRevealed[players[0]] != bytes32(0), "Player 0 has not revealed his choice");
        require(choicesRevealed[players[1]] != bytes32(0), "Player 1 has not revealed his choice");

        if(choicesRevealed[players[0]] == choicesRevealed[players[1]])
        {
            userBalances[players[1]] += 200;
            choicesRevealed[players[0]] = bytes32(0);
            choicesRevealed[players[1]] = bytes32(0);
            choices[players[0]] = bytes32(0);
            choices[players[1]] = bytes32(0);
        }
        else
        {
            userBalances[players[0]] += 200;
            choicesRevealed[players[0]] = bytes32(0);
            choicesRevealed[players[1]] = bytes32(0);
            choices[players[0]] = bytes32(0);
            choices[players[1]] = bytes32(0);
        }
    }
    receive() external payable {}
}
\end{verbatim}
\newpage

~\\

\end{questions}

\end{document}
