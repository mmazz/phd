\documentclass[12pt,addpoints,answers]{exam}
\usepackage[utf8]{inputenc}
\usepackage{amsmath,amsfonts,amssymb,amsthm}
\usepackage[margin=1in]{geometry}
\usepackage{mathtools}
\usepackage{hyperref}
\usepackage{fullpage}
\usepackage{microtype}
\usepackage{xspace}
\usepackage[svgnames]{xcolor}
\usepackage[sc]{mathpazo}
\usepackage{enumitem}
\usepackage{bm}

\pagestyle{head}

%----------Header--------------------%
\def\course{{\sc Foundations and Applications of Blockchains}}
\def\term{Depto. de Computaci\'{o}n, $2^{\textrm{do}}$ Cuatrimestre 2023}
\def\prof{Lecturer: Juan Garay}
\newcommand{\handout}[5]{
   \renewcommand{\thepage}{\arabic{page}}
   \begin{center}
   \framebox{
      \vbox{
    \hbox to 5.78in { \hfill \large{\course} \hfill }
    \vspace{2mm}
%    \hbox to 5.78in { \hfill \large{\prof} \hfill }
%       \vspace{2mm}
       \hbox to 5.78in { {\Large \hfill \textbf{#5}  \hfill} }
       \vspace{2mm}
       \hbox to 5.78in { \term \hfill \emph{#2}}
       \hbox to 5.78in { {#3 \hfill \emph{#4}}}
      }
   }
   \end{center}
   \vspace*{4mm}
}
\newcommand{\hw}[4]{\handout{#1}{#2}{#3}{#4}{Homework #1}}

% -- For ignoring stuff -- %
\newcommand{\ignore}[1]{}

\begin{document}

%----Specs: change accordingly-----%
\newif\ifstudent % comment out false
\studenttrue
% \studentfalse

\def\hwnum{1}
\def\issuedate{24/8/23}
\def\duedate{7/9/23, 3:30pm} %
\def\yourname{} % put your name here
%------------------------------%

\ifstudent
\hw{\hwnum}{\issuedate}{Student: \yourname}{Due: \duedate}%
\else
\hw{\hwnum}{\issuedate}{\prof}{Due: \duedate}%
\fi

% \ignore{

\noindent \textbf{Instructions}

\begin{itemize}
    \item Upload your solution to Campus; make sure it's only one file, and clearly write your name on the first page. Name the file \textsf{`$<$your last name$>$\_HW1.pdf.'}
    % {\bf Important:} Make sure to tap {\bf Turn in} after you upload your solution.

     If you are proficient with \LaTeX, you may also typeset your submission and submit in PDF format. To do so, uncomment the ``\%\textbackslash begin\{solution\}'' and ``\%\textbackslash end\{solution\}'' lines and write your solution between those two command lines.

      \item Your solutions will be graded on \emph{correctness} and
    \emph{clarity}. You should only submit work that you believe to be
    correct.
    % , and you will get significantly more partial credit if you     clearly identify the gap(s) in your solution.

    \item You may collaborate with others on this problem set.  However,
    you must \textbf{{write up your own solutions}} and \textbf{{list
      your collaborators and any external sources}} for each
    problem. Be ready to explain your solutions orally to a member of the course staff
    if asked.

    \ignore{
    \item For problems that require you to provide an algorithm, you must
    give a precise description of the algorithm, together with a proof
    of correctness and an analysis of its running time. You may use
    algorithms from class as subroutines. You may also use any facts
    that we proved in class or from the book.
    } %IGNORE

\end{itemize}

\noindent This homework contains \numquestions\ questions,
% \numpages\ pages
for a total of \numpoints \ points.
% and \numbonuspoints\ bonus points.

%\medskip
\newpage

%} %IGNORE

\begin{questions}


\question This question is about Merkle Trees (MTs).

\begin{parts}
    \part[3] Describe how (cryptographic) hash functions are used in a MT.

    \begin{solution} %Uncomment and type your solution here
        e divide al bloque en N partes iguales de una cierta cantidad de Bytes establecida, y se
        le aplica a cada uno la función Hash.
        Estos serán las hojas de nuestro árbol binario.
        Se concatenan luego de a pares consecutivos, es decir la hoja 0 con el 1, el 2 con el 3,
        y así sucesivamente y a esta concatenación se le aplica nuevamente la función de hash.
        Esto generara un nivel del árbol.
        De la misma manera se hace a este nivel lo hecho con las hojas, hasta llegar a la raíz, que
        terminara siendo un hash que involucra la información de todo el árbol.
    \end{solution}

    \part[3] Describe how a (complete, binary) MT is constructed for the following five chunks of data: ABC, DEF, GHI, KLM, OPQ.

    \begin{solution} %Uncomment and type your solution here
        De abajo hacia arriba.

        H(ABC), H(DEF), H(GHI), H(KLM), H(OPQ)\\
           H(H$_{ABC}|$H$_{DEF}$), H(H$_{GHI}|$H$_{KLM}$), H$_{OPQ}$\\
                H(H$_{ABCDEF}|$H$_{GHIKLM}$) , H$_{OPQ}$\\
                H(H$_{ABCDEFGHIKLM}|$ H$_{OPQ}$)
    \end{solution}

    \part[4] Describe how a Patricia Trie is constructed for the following key/value store: \{blah: 17, blahblah: 28, bored: 53, board: 39, aboard: 42, abroad: 17\}.
        \begin{solution} %Uncomment and type your solution here
            Primero tendremos la raíz, y agregamos la primer palabra, que sera \textit{blah}, que será en
            este momento un único nodo con valor 17, luego agregamos \textit{blahblah}, vemos que
            al moverse por el árbol va a encontrar las primeras 4 letras, y desde ese nodo (\textit{blah}) generara desde ahi otro
            nodo llamado igual con el valor 28.
            Luego al querer agregar \textit{bored}, al movernos por el arbol vemos que comparte solo la primer letra.
            De esta forma se separa el nodo \textit{blah} en un nodo \textit{b} que se desprende otro nodo
            \textit{lah}, manteniendo su hijo \textit{blah} y su valor 17, y ahora de este nuevo nodo,
            se genera otro llamado \textit{ored}, al cual se le pone el valor 53.

            Le sigue la palabra \textit{board}, al cual desde la raiz se mueve al nodo \textit{b}, y de ahi
            al nodo  \textit{ored}, compartiendo solo la primer letra, asi que se genera un nuevo
            split, quedando el nodo \textit{o} con el hijo \textit{red} con valor 53 y ahora se agrega
            otro hijo \textit{ard} con valor 39.

            Para la palabra \textit{aboard} vemos que desde la raiz nada empieza con la letra a asi que genera
            un hijo con dicho nombre y valor 42.
            Al querer agregar la nueva palabra \textit{abroad}, va al nodo anterior agregado, y genera
            un split, generando el nodo \textit{ab} con hijo \textit{oard} con valor 42, y agrega
            otro hijo con palabra \textit{road} de valor 17.
        \end{solution}
\end{parts}

\newpage

\question Cryptographic hash functions:

\begin{parts}

\part[5] Derive the formula for the {\em birthday paradox} (show your work, explaining every step) and
calculate the number of elements needed to find a collision with at least 50\%.

\begin{solution} %Uncomment and type your solution here
Para calcular la paradoja del cumpleaños, la cual es cuantas personas se requieren
para que dos cumplan en el mismo dia con una probabilidad mayor a 0.5 hago
basicamente hago 1 menos la productoria: $\prod_{i=0}^{k-1}  (n-k)/n^k$.
Donde n es la cantidad de posibilidades (en el caso del cumpleaños 365 dias) y k es
la cantidad de personas que se requieren para que se alcance el minimo valor superior a 0.5.
Para el caso de las colisiones, podemos hacer lo mismo pero esta vez n representaria
la cantidad de posibilidades de nuestro espacio de entrada.

    Esto se puede resolver numericamente o bien aproximar la productoria esa por $e^{-k^2/2n}$
    y para que uno menos esa exponencial de el minimo valor mayor a 0.5 se cumple cuando
    $k=\sqrt(n)$.
    Dando asi nuestra solucion.
\end{solution}

\part[5] Apply the above result to find out how many Bitcoin users are needed to initialize their wallet’s seed, which is based on a random selection of 12 random words from the list in \url{https://github.com/bitcoin/bips/blob/master/bip-0039/english.txt}, to have the event that, with probability at least 50\%, at least two users end up with exactly the same seed.

\begin{solution} %Uncomment and type your solution here
Para poder contestar tenemos quie saber cuantas semillas distintas tenemos.
Podemos ver que son 2048 palabras y formamos la semilla como la concatenacion de 12 de estas
de forma aleatoria.
Asumno que son 12 distintas.
Es decir tenemos que calcular la cantidad de semillas diferentes que podemos armar.
    Esto no es otra cosa que el combinatorio sin repeticion de (2048 12) = 11005261717918037175659349191168.
    Si ahora le tomamos la raiz nos daria la cantidad de usuarios que se requieren para tener una posibilidad
    de 0.5 de que tengan la misma semilla.
    Lo que nos da la absurda cifra de  3.3174179$\times 10^{15}$ usuarios.
\end{solution}

\end{parts}

\newpage

\question[10] Prof. Gray has designed a cryptographic hash function $H_G : \{0, 1\}^\ast \rightarrow \{0, 1\}^n$. One of his brilliant ideas is to make $H_G(x) = x$ if $x$ is an $n$-bit string (assume the behavior of $H_G$ is much more complicated on inputs of other lengths). That way, we know with certainty that there are no collisions among $n$-bit strings. Has Prof. Gray made a good design decision? Justify your answer, in particular listing the properties that may or may not be satisfied by the design.

\begin{solution} %Uncomment and type your solution here
Una de las propiedades que esta funcion hash no cumple es la distribucion amplia
del input.
En particular en este caso cuando el input es de $n-bit$ es 1 a 1 y esto genera
que un pequeño cambio en la entrada genere un pequeño cambio en la salida.
Al tener esta propiedad, es fácil de invertir, dada la salida sabemos la entrada.
\end{solution}

\newpage

\question[10] As we saw in class, the main security notion for digital signatures is called {\em existential unforgeability}, which prevents an attacker from producing a signature on a message that has not been produced by the legitimate signer. I.e., the attacker should not be able to produce the pair $(m, \sigma)$, where $\sigma$ was not produced by the legitimate signer.

Show an attack on the plain RSA signature scheme we saw in class in which an attacker forges a signature on an arbitrary message $m$ by asking the signer to sign two other different messages (not necessarily unrelated to $m$).

\begin{solution} %Uncomment and type your solution here
Esto es posible gracias a que el esquema RSA tiene la propiedad de que la multiplicación es
homomórfica.
Esto significa que descencriptar la multiplicación de dos mensajes encriptados nos
devuelve el mismo resultado que los mensajes sin encriptar multiplicados.

    Con esto, un atacante puede firmar un mensaje diferente pidiendo (u obteniendo)
la firma de dos mensajes $m_1$ y $m_2$, teniendo así sus firmas $s_1$ y $s_2$ respectivamente.
Teniendo el parámetro $n$ de la llave publica del que firma, el atacante puede producir
un mensaje nuevo  $m = m_1*m_2 \text{ mod }n$ con firma $s = s_1*s_2\text{ mod }n$.
Luego esto puede ser subido con la llave publica original, verificando que $s$ es una
firma valida para $m$ para dicho juego de llaves.

Para entender el por que funciona, primero  acordémonos que $s_i = m_i^d \text{ mod }n$, con $d$ parte
de la llave secreta.
Y la verificación es usando el parámetro $e$ de la llave publica $s_i^e \text{ mod }n = m_i$.
Entonces nuestro firma forjada $s=(m_1*m_2)^d \text{ mod }n$ y entonces al verificarlo es
    elevarlo a la $e$, $verificacion(s)=(m_1*m_2)^{d*e} \text{ mod }n$, pero como $d*e$ es congruente a
    $1\text{ mod }n$, podemos reemplazarlo con 1, quedándonos $verificacion(s)=(m_1*m_2)^{1} \text{ mod }n = m_1*m_2  = m$.
    Que era lo que queríamos probar.
\end{solution}

\newpage

\question[10] A Bitcoin miner creates a block $B$ which contains address $\alpha$, on which it wants to receive its rewards. An attacker changes the contents of $B$, such that instead of $\alpha$ it defines a new address $\alpha'$, which is controlled by the attacker. Will the attacker receive the rewards that the miner tries to claim? Why or why not? Give a detailed explanation of your answer.

\begin{solution} %Uncomment and type your solution here
Al generar un cambio de en el bloque B, por el atacante, dicho bloque tendrá un valor
de hash completamente diferente, generando que no cumpla con la condición encontrada por
el minero para que dicho bloque satisfaga la prueba de trabajo, un hash con una $x$
cantidad de ceros al comienzo.
Al suceder eso el bloque no sera aceptado y por ende el atacante no recibirá la recompensa.
\end{solution}

\newpage

\question[10] Using the course’s Goerli  Testnet chain, send 1 ETH to the address of a fellow student. Describe how you conducted the payment, including the transaction’s id and addresses which you used. (Refer to the instructions from Lecture 5 on ``How to Connect to a Public Testnet'' as well as the Solidity tutorial material and pointers.)

%\begin{solution} %Uncomment and type your solution here

%\end{solution}

\newpage

\question The following smart contract has been deployed on the Goerli Testnet chain:

{\footnotesize

\begin{verbatim}
// SPDX-License-Identifier: GPL-3.0

pragma solidity >=0.7.0 <0.9.0;

contract Bank {
    mapping(address => uint256) balance;
    address[] public customers;

    event Deposit(address customer, string message);
    event Withdrawal(address customer);

    function deposit(string memory message) public payable {
        require(msg.value > 10);
        balance[msg.sender] += msg.value - 10;

        customers.push(msg.sender);

        emit Deposit(msg.sender, message);
    }

    function withdraw() public {
        uint256 b = balance[msg.sender];
        balance[msg.sender] = 0;
        payable(msg.sender).transfer(b);

        emit Withdrawal(msg.sender);
    }

    function getBalance() public view returns (uint256) {
        return balance[msg.sender];
    }

    function empty() public {
        require(msg.sender == customers[0]);

        payable(msg.sender).transfer(address(this).balance);
    }
}
\end{verbatim}

} %FOOTNOTESIZE

 Its deployed address is: 0x4310a5f4Bb3b4Ff09f0694218fA50D2767a00b86. You can compile it and interact with it using Remix. You should successfully create a transaction that interacts with the contract, either depositing or withdrawing from it some coins.

 \begin{parts}

 \part[5] Describe the contract’s functionality (that is, the purpose of each variable, function, and event).

%\begin{solution} %Uncomment and type your solution here

%\end{solution}

\part[15]

 Provide the id of the transaction you performed and the address you used.

 %\begin{solution} %Uncomment and type your solution here

%\end{solution}

\end{parts}

\newpage

~\\

\end{questions}

\end{document}
