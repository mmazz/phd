\documentclass[12pt,addpoints,answers]{exam}
\usepackage[utf8]{inputenc}
\usepackage{amsmath,amsfonts,amssymb,amsthm}
\usepackage[margin=1in]{geometry}
\usepackage{mathtools}
\usepackage{hyperref}
\usepackage{fullpage}
\usepackage{microtype}
\usepackage{xspace}
\usepackage[svgnames]{xcolor}
\usepackage[sc]{mathpazo}
\usepackage{enumitem}
\usepackage{bm}

\pagestyle{head}

% TIRET DEFINITIONS

\newlength{\saveparindent}
\setlength{\saveparindent}{\parindent}
\newlength{\saveparskip}
\setlength{\saveparskip}{\parskip}
\newcounter{ctr}
\newcounter{savectr}
\newcounter{ectr}
\newcounter{sctr}


\newenvironment{tiret}{%
\begin{list}{\hspace{2pt}\rule[0.5ex]{6pt}{1pt}\hfill}{\labelwidth=15pt%
\labelsep=5pt \leftmargin=20pt \topsep=3pt%
\setlength{\listparindent}{\saveparindent}%
\setlength{\parsep}{\saveparskip}%
\setlength{\itemsep}{0pt} }}{\end{list}}


%----------Header--------------------%
\def\course{{\sc Fundamentos y Aplicaciones de Blockchains}}
\def\term{Depto. de Computaci\'{o}n, $2^{\textrm{do}}$ Cuatrimestre 2023}
\def\prof{Profesor: Juan Garay}
\newcommand{\handout}[5]{
   \renewcommand{\thepage}{\arabic{page}}
   \begin{center}
   \framebox{
      \vbox{
    \hbox to 5.78in { \hfill \large{\course} \hfill }
    \vspace{2mm}
%    \hbox to 5.78in { \hfill \large{\prof} \hfill }
%       \vspace{2mm}
       \hbox to 5.78in { {\Large \hfill \textbf{#5}  \hfill} }
       \vspace{2mm}
       \hbox to 5.78in { \term \hfill \emph{#2}}
       \hbox to 5.78in { {#3 \hfill \emph{#4}}}
      }
   }
   \end{center}
   \vspace*{4mm}
}
\newcommand{\hw}[4]{\handout{#1}{#2}{#3}{#4}{Homework #1}}

% -- For ignoring stuff -- %
\newcommand{\ignore}[1]{}

\begin{document}

%----Specs: change accordingly-----%
\newif\ifstudent % comment out false
\studenttrue
% \studentfalse

\def\hwnum{3}
\def\issuedate{25/9/23}
\def\duedate{9/10/23, 20:00hs} %
\def\yourname{} % put your name here
%------------------------------%

\ifstudent
\hw{\hwnum}{\issuedate}{Student: Matias Mazzanti}{Due: \duedate}%
\else
\hw{\hwnum}{\issuedate}{\prof}{Due: \duedate}%
\fi

% \ignore{

\noindent \textbf{Instructions}

\begin{itemize}
    \item Upload your solution to Campus; make sure it's only one file, and clearly write your name on the first page. Name the file \textsf{`$<$your last name$>$\_HW3.pdf.'}
    % {\bf Important:} Make sure to tap {\bf Turn in} after you upload your solution.

     If you are proficient with \LaTeX, you may also typeset your submission and submit in PDF format. To do so, uncomment the ``\%\textbackslash begin\{solution\}'' and ``\%\textbackslash end\{solution\}'' lines and write your solution between those two command lines.

      \item Your solutions will be graded on \emph{correctness} and
    \emph{clarity}. You should only submit work that you believe to be
    correct.
    % , and you will get significantly more partial credit if you     clearly identify the gap(s) in your solution.

    \item You may collaborate with others on this problem set.  However,
    you must \textbf{{write up your own solutions}} and \textbf{{list
      your collaborators and any external sources}} for each
    problem. Be ready to explain your solutions orally to a member of the course staff
    if asked.

    \ignore{
    \item For problems that require you to provide an algorithm, you must
    give a precise description of the algorithm, together with a proof
    of correctness and an analysis of its running time. You may use
    algorithms from class as subroutines. You may also use any facts
    that we proved in class or from the book.
    } %IGNORE

\end{itemize}

\noindent This homework contains \numquestions\ questions,
% \numpages\ pages
for a total of \numpoints \ points.
% and \numbonuspoints\ bonus points.

%\medskip
\newpage

%} %IGNORE

\begin{questions}


\question \textbf{Bitcoin transactions:}
         \begin{parts}
                   \part[7] Describe the mechanism used in the actual Bitcoin application that  the miners follow in order to insert transactions into a block. Would the Consistency (aka Persistence) and Liveness properties we saw in class, as well as the $V,I,R$ functions have to be modified to capture the actual mechanism? Elaborate.

     %\begin{solution} %Uncomment and type your solution here

    %\end{solution}

            \part[3] Describe the purpose of a {\em coinbase} transaction.

     \begin{solution} %Uncomment and type your solution here
        El objetivo es tener un sistema de pago a los mineros que son los que mantienen el estado de la Blockchain.
         Entonces el minero cuando hace la POW tiene agregada esta transaccion hacia el para tener el pago de dicho trabajo.
         Esa critpomoneda que se agregan como pago es nueva, de ahi el termino minar.
         Es la forma que tienen esas criptomonedas de generar (''imprimir'') mas.
    \end{solution}

            \end{parts}


\newpage

\question \textbf{Proofs of work:}

\begin{parts}

\part[5] Describe the purpose and implementation of the {\bf 2x1 PoW} technique.

     \begin{solution} %Uncomment and type your solution here
         \begin{itemize}
            \item El proposito detras de esta tecnica es generar un costo al agregado de transacciones en un bloque.
                 Por que una vez que un minero corrupto resuelve una proof of work, les es gratis poner en la parte de datos todo lo que quieren.
                 Generando que no se cumpla la Chain Quality.
                 Entonces la idea es generar un costo a esto.
                 Esto se logra que despues de resolver la primer POW, se tenga que resolver otra POW para meter transacciones/datos.
                 Entonces el numero de bloques termina siendo proporcional no solo al poder de hash, si no que tambien va a ser proporcional al numero de transacciones.
                 Con esto se logra que el concenso para el acuerdo Bizantino sea optimo permitiendo que se pueda permitir que los usuarios corruptos
                 sean menos de la mitad y no menos de un tercio.
             \item
         \end{itemize}

    \end{solution}

\part[7] Let $T_1$ and $T_2$ be the target values in 2x1 PoW, and $\kappa$ the size of the hash function's output. What relation should they satisfy for the technique to work? Elaborate.

    %\begin{solution} %Uncomment and type your solution here

    %\end{solution}

  \part[8] Design and argue correctness of an $\ell$x1 PoW scheme. Note that such a scheme would enable an $\ell$-parallel blockchain. What properties of the blockchain or ledger application would, if any,  benefit from a parallel blockchain? Elaborate.

   %\begin{solution} %Uncomment and type your solution here

    %\end{solution}

    \end{parts}

\newpage

\question \textbf{Strong consensus:} We saw how the Bitcoin backbone protocol can be used to solve the {\em consensus} problem (aka {\em Byzantine agreement}). In the {\em strong consensus} problem, the {\em Validity} condition is strengthened to require that the output value be one of the honest parties' inputs---this property is called {\em Strong Validity}. (Note that this distinction is relevant only in the case of non-binary inputs.)

\begin{parts}

\part[5] What should be the assumption on the adversarial computational power (similar to the ``Honest Majority Assumption'') for Strong Validity to hold?

    %\begin{solution} %Uncomment and type your solution here

    %\end{solution}

\part[5]  State and prove the Strong Validity lemma.

    %\begin{solution} %Uncomment and type your solution here

    %\end{solution}

\end{parts}

\newpage

\question \textbf{Smart contract programming:}  In this assignment you will create your own custom token. Your contract should implement the public API described below:

\begin{itemize}
\item \underline{\textbf{owner:}} a public payable address that defines the contract’s ``owner,'' that is, the user that deploys the contract
\item \textbf{\textit{Transfer(address indexed from, address indexed to, uint256 value):}} an event that contains two addresses and a uint256
\item \textbf{\textit{Mint(address indexed to, uint256 value):}} an event that contains an address and a uint256
\item \textbf{\textit{Sell(address indexed from, uint256 value):}} an event that contains an address and a uint256

\item \textbf{totalSupply():} a view function that returns a uint256 of the total amount of minted tokens
\item \textbf{balanceOf(address\_account):} a view function returns a uint256 of the amount of tokens
an address owns
\item \textbf{getName():} a view function that returns a string with the token’s name
\item \textbf{getSymbol():} a view function that returns a string with the token’s symbol
\item \textbf{getPrice():} a view function that returns a uint128 with the token’s price (at which users can
redeem their tokens)

\item \textbf{transfer(address to, uint256 value):} a function that transfers \emph{value} amount of tokens
between the caller’s address and the address to; if the transfer completes successfully, the
function emits an event \emph{Transfer} with the sender’s and receiver’s addresses and the amount
of transferred tokens and returns a boolean value (\emph{true})
\item \textbf{mint(address to, uint256 value):} a function that enables \emph{only the owner} to create value new tokens and give them to address to; if the operation completes successfully, the
function emits an event \emph{Mint} with the receiver’s address and the amount of minted tokens
and returns a boolean value (\emph{true})
\item \textbf{sell(uint256 value):} a function that enables a user to sell tokens for wei at a price of \emph{600 wei per token}; if the operation completes successfully, the sold tokens are removed from
the circulating supply, and the function emits an event \emph{Sell} with the seller’s address and the
amount of sold tokens and returns a boolean value (\emph{true})
\item \textbf{close():} a function that enables \emph{only the owner} to destroy the contract; the contract’s
balance in wei, at the moment of destruction, should be transferred to the owner's address
\item \textbf{fallback} functions that enable anyone to send Ether to the contract’s account
\item \textbf{constructor} function that initializes the contract as needed

\end{itemize}

You should implement the smart contract and deploy it on the course's Goerli Testnet. Your contract should be as secure and gas efficient as possible. After deploying your contract, you should buy, transfer, and sell a token in the contract.

Your contract should implement the above API \textbf{exactly as specified}. \emph{Do not} omit implementing one of the above variables/functions/events, do not change their name or parameters, and do not add other public variables/functions. You can define other private/internal functions/variables, if necessary.

You should provide:

\begin{parts}

\part[5] A detailed description of your high-level design decisions, including (but not limited to):
\begin{tiret}
\item  What internal variables did you use?
\item What is the process of buying/selling tokens and changing the price?
\item How can users access their token balance?
\item How did you link the library to your contract?
\end{tiret}

    \begin{solution} %Uncomment and type your solution here
        Para este contrato uso dos diccionarios, uno para tener la cantidad de plata que puso cada usiario en ethers y otra
        para tener la cantidad de tokens que tiene cada uno.
        Despues tengo una string para el nombre del token y otro para el de su symbolo.
        Y por ultimo tengo 3 enteros, uno para el precio de token en wei, otro para la cantidad posible de emicion de tokens y el ultimo la cantidad de
        tokens emitidos.

        El usuario envia ethers al contrato y de aca el owner puede emitirle tokens.
        Para vender el usuario simplemente pone cuantos tokens quiere vender.

        Para saber su balance simplemente tiene que poner el address suyo en balanceOf y listo.
    \end{solution}

\part[5] A detailed gas evaluation of your implementation, including:
\begin{tiret}
\item The cost of deploying and interacting with your contract.
\item Techniques to make your contract more cost effective.
\item What was the gas impact of using the deployed library instance, compared to
including its code in your contract?
\end{tiret}

    \begin{solution} %Uncomment and type your solution here
        El del contracto cuesta 0.003206567508978389 ETH y el interactuar 0.000108640000304192 ETH.
        Para hacerlo lo mas eficientemente posible en cuanto a gas es usar una estrucutra
        de datos como los diccionarios para evitar cualquier tipo de loop a la hora de
        buscar los balances y demas.
    \end{solution}

\part[5] A thorough listing of potential hazards and vulnerabilities that can occur in the smart
contract and a detailed analysis of the security mechanisms that can mitigate these
hazards.

    \begin{solution} %Uncomment and type your solution here
        Los Hazards posibles que se me ocurren pueden ser a la hora de mover tokens
    \end{solution}

\part[5] The transaction history of the deployment of and interaction with your contract.

    \begin{solution} %Uncomment and type your solution here
 \begin{itemize}
    \item Deploy: 0x11546dd3f3140363ca1a3ed223572120fedab35e14b71a55bb085f301d7830c9
    \item fallBack: 0x312f48ea06a719105ea44ace9cf0fbed305f4a2bd4f8934d130d7b941f379219
    \item Mint 1 token: 0xf596013730a6df3e07d562194b36b34f4763f01a5ab7b18a8f62179aa08eeab3
    \item Sell: 0x117cce502f11df877d53ef116ac614b15e7e3815d4a71bb57e71894411026743
    \item Close: 0xba6d985dd60ccc1ca475e06dab42508b8aac05dce46ffd9ad8b4579625724b94
 \end{itemize}
    \end{solution}

\part[5] The code of your contract.

    \begin{solution} %Uncomment and type your solution here

\begin{verbatim}

// SPDX-License-Identifier: GPL-3.0
pragma solidity ^0.8.0;


contract Token
{
    // a public payable address that defines the contract’s “owner,” that is, the
    // user that deploys the contract
    address payable public owner;
    string private _symbol;
    string private _name;
    uint256 private _totalSupply = 0;
    uint256 private _maxTokens = 1_000_000;
    uint128 private _price = 600;
    mapping(address => uint256) balances;
    mapping(address => uint256) paid;


    // an event that contains two addresses and a uint256
    event Transfer(address indexed from, address indexed to, uint256 value);
    // an event that contains an address and a uint256
    // emitir seria
    event Mint(address indexed to,  uint256 value);

       // an event that contains an address and a uint256
    event Sell(address indexed from, uint256 value);


    // function that initializes the contract as needed
    constructor()
    {
        _symbol = "CC";
        _name = "Chiquito Coin";
        owner = payable(msg.sender);
      //  balances[owner] = _totalSupply;
        emit Transfer(address(0), owner, _totalSupply);
    }




    // a view function that returns a uint256 of the total amount of minted tokens
    function totalSupply() external view returns (uint256)
    {
        return _totalSupply;
    }

    // a view function returns a uint256 of the amount of tokens an address owns
    function balanceOf(address account) external view returns (uint256 balance)
    {
        return balances[account];
    }

    //a view function that returns a string with the token’s name
    function getName() public view returns (string memory)
    {
        return _name;
    }

    // a view function that returns a string with the token’s symbol
    function getSymbol()  public view returns  (string memory)
    {
        return _symbol;
    }

    // a view function that returns a uint128 with the token’s price (at which
    // users can redeem their tokens)
    function getPrice() public view returns  (uint128)
    {
        return _price;
    }

    //a function that transfers value amount of tokens between the caller’s address and
    // the address to; if the transfer completes successfully, the function emits an event
    //  Transfer with the sender’s and receiver’s addresses and the amount of transferred
    // tokens and returns a boolean value (true)
    function transfer(address to, uint256 value)  public returns (bool success)
    {
        require(to != address(0), "ERC20: transfer to the zero address");
        require(value <= balances[msg.sender], "ERC20: Insufficient balance");

        balances[msg.sender] -= value;
        balances[to] += value;
        emit Transfer(msg.sender, to, value);
        return true;
    }

    // a function that enables only the owner to create value new tokens and give them
    // to address to; if the operation completes successfully, the function emits an
    // event Mint with the receiver’s address and the amount of minted tokens and returns
    // a boolean value (true)
    function mint(address to, uint256 value) public returns(bool success)
    {
        require(to != address(0), "Mint to the zero address");
        //require(paid[to] >= value * _price);
        require(msg.sender == owner, "Do not have privilages to mint");
        require(_totalSupply+value <= _maxTokens, "No more tokens to emit");
        require(paid[to] >= value * _price);
        paid[to] -= value * _price;
        _totalSupply += value;
        balances[to] += value;
        emit Mint( to, value);
        return true;
    }

    // a function that enables a user to sell tokens for wei at a price of 600 wei per token;
    // if the operation completes successfully, the sold tokens are removed from the
    // circulating supply, and the function emits an event Sell with the seller’s address
    // and the amount of sold tokens and returns a boolean value (true)
    function sell(uint256 value) public returns(bool)
    {
        require(balances[msg.sender]>= value, "Not enough tokens");
        balances[msg.sender] -= value;
        _totalSupply -= value;
        payable(msg.sender).transfer(value*_price);
        emit Sell(msg.sender, value);
        return true;
    }

    // a function that enables only the owner to destroy the contract; the
    // contract’s balance in wei, at the moment of destruction, should be transferred to
    // the owner’s address
    function close() public
    {
        require(msg.sender == owner, "Do not have privilages to close contract");
        selfdestruct(owner);
    }

    //functions that enable anyone to send Ether to the contract’s account
    receive() external payable {
        paid[msg.sender] += msg.value;
    }
}
\end{verbatim}
    \end{solution}

\end{parts}

\newpage


~\\

\end{questions}

\end{document}
