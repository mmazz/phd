\documentclass[12pt, oneside]{article}
\usepackage{amsmath, amsthm, amssymb, calrsfs, wasysym, verbatim,  color, graphics, geometry}
\usepackage{hyperref}
\geometry{tmargin=.75in, bmargin=.75in, lmargin=.75in, rmargin = .75in}
\usepackage[colorinlistoftodos,color=cyan]{todonotes}
\newcommand{\Hc}{\mathcal{H}}
\newcommand{\Rc}{\mathcal{R}}
\newcommand{\R}{\mathbb{R}}
\newcommand{\C}{\mathbb{C}}
\newcommand{\Z}{\mathbb{Z}}
\newcommand{\N}{\mathbb{N}}
\newcommand{\Q}{\mathbb{Q}}
\newcommand{\Cdot}{\boldsymbol{\cdot}}

\newtheorem{thm}{Theorem}
\newtheorem{defn}{Definition}
\newtheorem{conv}{Convention}
\newtheorem{rem}{Remark}
\newtheorem{lem}{Lemma}
\newtheorem{cor}{Corollary}


\title{Resume: }
\author{Matias Mazzanti}
\date{}

\begin{document}

\maketitle
\tableofcontents

\section{Avances}

\subsection{RNS}

Sea $\bold{r}$ los residuos de un valor $x$ modulo $q = (q_1, ...,q_L)$.
Donde ademas tenemos $Q = \prod_0^L q_i$, $Q_i = \frac{Q}{q_i}$ y ademas $\left(\frac{1}{Q_{i}}\right)mod q_i$ es la inversa multiplicativa.

Entonces tenemos que:
\begin{equation}
    x = \left(\sum_0^L r_i\left[\left(\frac{1}{Q_{i}}\right)mod q_i\right]mod q_i \times Q_i\right) mod Q
\end{equation}

Es decir que si llamamos a todo lo que esta dento del paréntesis $R$ y a los residuos le agregamos un error $e$, en general donde e es todo cero excepto un elemento.
Tenemos que
\begin{equation}
    x = R + e mod Q = \left(R mod Q + e mod Q \right) mod Q = \left( x + e mod Q \right) mod Q
\end{equation}
\end{document}
% https://bit-ml.github.io/blog/post/homomorphic-encryption-toy-implementation-in-python/

