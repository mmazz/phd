\documentclass[12pt, oneside]{article}
\usepackage{amsmath, amsthm, amssymb, calrsfs, wasysym, verbatim,  color, graphics, geometry}
\usepackage{hyperref}
\usepackage{amsmath}
\geometry{tmargin=.75in, bmargin=.75in, lmargin=.75in, rmargin = .75in}
\usepackage[colorinlistoftodos,color=cyan]{todonotes}
\newcommand{\Hc}{\mathcal{H}}
\newcommand{\Rc}{\mathcal{R}}
\newcommand{\R}{\mathbb{R}}
\newcommand{\C}{\mathbb{C}}
\newcommand{\Z}{\mathbb{Z}}
\newcommand{\N}{\mathbb{N}}
\newcommand{\Q}{\mathbb{Q}}
\newcommand{\Cdot}{\boldsymbol{\cdot}}

\newtheorem{thm}{Theorem}
\newtheorem{defn}{Definition}
\newtheorem{conv}{Convention}
\newtheorem{rem}{Remark}
\newtheorem{lem}{Lemma}
\newtheorem{cor}{Corollary}


\title{Resume: }
\author{Matias Mazzanti}
\date{}

\begin{document}

\maketitle
\tableofcontents

\section{Avances}

\subsection{RNS}

Sea $\bold{r} = (r_1, ..., r_L)$ los residuos de un valor $x$ modulo $q = (q_1, ...,q_L)$.
Donde ademas tenemos $Q = \prod_0^L q_i$, $Q_i = \frac{Q}{q_i}$ y ademas $\left(\frac{1}{Q_{i}}\right)\text{ mod } q_i$ es la inversa multiplicativa de $Q_i$.

Entonces tenemos que:
\begin{equation}
    x = \left(\sum_{i=0}^L r_i\left[\left(\frac{1}{Q_{i}}\right)\text{ mod } q_i\right]\text{ mod } q_i \times Q_i\right) \text{ mod } Q
\end{equation}

Es decir que si le agregamos un error $e_i$ a los residuos tenemos algo del estilo:

\begin{equation}
    x = \left(\sum_{i=0}^L (r_i+e_i)\underbrace{\left[\left(\frac{1}{Q_{i}}\right)\text{ mod } q_i\right]}_{Q_i^{-1}}\text{ mod } q_i \times Q_i\right) \text{ mod } Q
\end{equation}

\begin{equation}
    x = \left(\sum_{i=0}^L \left[(r_i+e_i)\times Q_i^{-1}\right]\text{ mod } q_i \times Q_i\right) \text{ mod } Q
\end{equation}

\begin{equation}
    \begin{split}
        x &= \left(\sum_{i=0}^L \left[(r_i\times Q_i^{-1} +e_i\times Q_i^{-1})\right]\text{ mod } q_i \times Q_i\right) \text{ mod } Q \\
          &= \left(\sum_{i=0}^L \left[r_i\times Q_i^{-1} \text{ mod } q_i +e_i\times Q_i^{-1}\text{ mod } q_i \right]\text{ mod } q_i \times Q_i\right) \text{ mod } Q \\
    \end{split}
\end{equation}

\subsection{FFT}

Supongamos el caso mas sencillo que es tener un DFT, es decir la transformada no es mas que una
multiplicación matricial.
En este caso no estoy considerando ni las constantes de normalización ni la utilización de bit reversed order que
creo que solo cambiaría el orden de las columnas de la matriz en cuestión.

Es decir dado mi vector $x$, al aplicarle la matriz de Vandermonde $W$ obtengo $FFT(x)$.
Es decir $ y = W \times x$.
Si a esta $y$ le introduzco un error $e$, tendría $z = y + e = W\times x + e$.
Al aplicarle la matriz inversa para volver al espacio original.
\begin{equation*}
    x' = W^{-1}\times z = W^{-1}\times (y+e) = W^{-1}\times y + W^{-1}\times e = x + W^{-1}\times e
\end{equation*}

Es decir que al calcular el error entre $x$ y $x'$, tenemos:
$\lVert x-x' \rVert = \lVert W^{-1} \times e \rVert$

\subsection {NTT}
La NTT también es pensada como una multiplicación matricial así que debería  ser igual que FFT.

\subsection{RNS + FFT}
Al modificar codificación primero tendría que hacer la inversa de la FFT ( o NTT), y de ahi calcular la
inversa de RNS.
Es decir lo que estoy modificando son los residuos y a eso le aplico la inversa de RNS.

Entonces dado mi vector $r$ modificado, lo que me queda al aplicarle IFFT es  $r' = r + W^{-1}\times e$.
Aunque $e$ sea todo cero excepto un elemento i-esimo, al ser una multiplicación matricial,
estaría afectando a todas las componentes de $r$.
Cada componente $r'_i$ tendrá un error $e'_i = (W^{-1}\times e)_i$.
La cuestión es como inserto este error a RNS y tener una expresión analítica que me sirva.



\end{document}
% https://bit-ml.github.io/blog/post/homomorphic-encryption-toy-implementation-in-python/

